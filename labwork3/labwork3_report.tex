


\documentclass[a4paper,12pt]{article}
\usepackage{graphicx}
\usepackage{amsmath}
\usepackage{hyperref}

\title{Labwork 3 Report}
\author{Do Thanh Dat - M23.ICT.002}
\date{\today}

\begin{document}

\maketitle

\section*{Introduction}
This report details the implementation of logistic regression from scratch using gradient descent optimization. The input data is a CSV file containing three columns:\\
\textbf{Experience}, \textbf{Salary}, and \textbf{Loan}, where \textbf{Loan} is the target variable. The goal is to predict whether a loan is approved (1) or rejected (0) based on the given features.

\section*{Methodology}
The logistic regression model was implemented as follows:
\begin{enumerate}
    \item \textbf{Data Loading}: The CSV file was read, and the features and target variable were extracted.
    \item \textbf{Data Preparation}: A bias term was added to the features to enable proper weight optimization.
    \item \textbf{Model Definition}: The sigmoid function was used to compute probabilities.
    \item \textbf{Loss Function}: Binary cross-entropy loss was computed to evaluate the model's performance.
    \item \textbf{Gradient Descent}: Weights were updated iteratively using the gradient of the loss function with respect to the weights.
\end{enumerate}

\section*{Results}
The training process resulted in the following weights:
\begin{itemize}
    \item $w_0$ (bias): {-3.6258}
    \item $w_1$ (Experience): {2.2566}
    \item $w_2$ (Salary): {0.1947}
\end{itemize}

Intermediate loss values during training:
\begin{itemize}
    \item Epoch 0: Loss = {0.6931}
    \item Epoch 100: Loss = {0.4732}
    \item Epoch 200: Loss = {0.4431}
    \item Epoch 300: Loss = {0.4218}
    \item Epoch 400: Loss = {0.4037}
    \item Epoch 500: Loss = {0.3877}
    \item Epoch 600: Loss = {0.3736}
    \item Epoch 700: Loss = {0.3612}
    \item Epoch 800: Loss = {0.3501}
    \item Epoch 900: Loss = {0.3402}
    \item Epoch 999: Loss = {0.3314}
\end{itemize}


\section*{Conclusion}
The logistic regression model was successfully implemented, and the weights were optimized using gradient descent. This implementation provides a foundation for further exploration of classification problems and demonstrates the mathematical basis of logistic regression.



\end{document}
